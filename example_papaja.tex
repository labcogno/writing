\documentclass[english,floatsintext,man]{apa6}

\usepackage{amssymb,amsmath}
\usepackage{ifxetex,ifluatex}
\usepackage{fixltx2e} % provides \textsubscript
\ifnum 0\ifxetex 1\fi\ifluatex 1\fi=0 % if pdftex
  \usepackage[T1]{fontenc}
  \usepackage[utf8]{inputenc}
\else % if luatex or xelatex
  \ifxetex
    \usepackage{mathspec}
    \usepackage{xltxtra,xunicode}
  \else
    \usepackage{fontspec}
  \fi
  \defaultfontfeatures{Mapping=tex-text,Scale=MatchLowercase}
  \newcommand{\euro}{€}
\fi
% use upquote if available, for straight quotes in verbatim environments
\IfFileExists{upquote.sty}{\usepackage{upquote}}{}
% use microtype if available
\IfFileExists{microtype.sty}{\usepackage{microtype}}{}

% Table formatting
\usepackage{longtable, booktabs}
\usepackage{lscape}
% \usepackage[counterclockwise]{rotating}   % Landscape page setup for large tables
\usepackage{multirow}		% Table styling
\usepackage{tabularx}		% Control Column width
\usepackage[flushleft]{threeparttable}	% Allows for three part tables with a specified notes section
\usepackage{threeparttablex}            % Lets threeparttable work with longtable

% Create new environments so endfloat can handle them
% \newenvironment{ltable}
%   {\begin{landscape}\begin{center}\begin{threeparttable}}
%   {\end{threeparttable}\end{center}\end{landscape}}

\newenvironment{lltable}
  {\begin{landscape}\begin{center}\begin{ThreePartTable}}
  {\end{ThreePartTable}\end{center}\end{landscape}}




% The following enables adjusting longtable caption width to table width
% Solution found at http://golatex.de/longtable-mit-caption-so-breit-wie-die-tabelle-t15767.html
\makeatletter
\newcommand\LastLTentrywidth{1em}
\newlength\longtablewidth
\setlength{\longtablewidth}{1in}
\newcommand\getlongtablewidth{%
 \begingroup
  \ifcsname LT@\roman{LT@tables}\endcsname
  \global\longtablewidth=0pt
  \renewcommand\LT@entry[2]{\global\advance\longtablewidth by ##2\relax\gdef\LastLTentrywidth{##2}}%
  \@nameuse{LT@\roman{LT@tables}}%
  \fi
\endgroup}


\ifxetex
  \usepackage[setpagesize=false, % page size defined by xetex
              unicode=false, % unicode breaks when used with xetex
              xetex]{hyperref}
\else
  \usepackage[unicode=true]{hyperref}
\fi
\hypersetup{breaklinks=true,
            pdfauthor={},
            pdftitle={Example for Converting RMarkdown into Word Doc and PDF with Papaja},
            colorlinks=true,
            citecolor=blue,
            urlcolor=blue,
            linkcolor=blue,
            pdfborder={0 0 0}}
\urlstyle{same}  % don't use monospace font for urls

\setlength{\parindent}{0pt}
%\setlength{\parskip}{0pt plus 0pt minus 0pt}

\setlength{\emergencystretch}{3em}  % prevent overfull lines

\ifxetex
  \usepackage{polyglossia}
  \setmainlanguage{}
\else
  \usepackage[english]{babel}
\fi

% Manuscript styling
\captionsetup{font=singlespacing,justification=justified}
\usepackage{csquotes}
\usepackage{upgreek}

 % Line numbering
  \usepackage{lineno}
  \linenumbers


\usepackage{tikz} % Variable definition to generate author note

% fix for \tightlist problem in pandoc 1.14
\providecommand{\tightlist}{%
  \setlength{\itemsep}{0pt}\setlength{\parskip}{0pt}}

% Essential manuscript parts
  \title{Example for Converting RMarkdown into Word Doc and PDF with Papaja}

  \shorttitle{RMarkdown \& Papaja}


  \author{Monica Li\textsuperscript{a}~\& Imaginary Author Two\textsuperscript{b}}

  \def\affdep{{"", ""}}%
  \def\affcity{{"", ""}}%

  \affiliation{
    \vspace{0.5cm}
          \textsuperscript{a} University of Connecticut\\
          \textsuperscript{b} Institute of Somewhere Over the Rainbow  }

 % If no author_note is defined give only author information if available
      \newcounter{author}
                              \authornote{
            Correspondence concerning this article should be addressed to Monica Li, Psychological Sciences Department, 406 Babbidge Road, Unit 1020, Storrs,
CT 06269. E-mail: \href{mailto:monica.yc.li@gmail.com}{\nolinkurl{monica.yc.li@gmail.com}}
          }
                                  

  \abstract{This document has similar content as \texttt{example\_pandoc.md} but
with more sophisticated format settings for the APA citation style. For
more detailed examples including R syntax, check out \texttt{papaja}'s
\href{https://github.com/crsh/papaja}{GitHub repo}.}
  \keywords{APA style, knitr, R, R markdown, papaja, docx, pdf \\

    \indent Word count: {[}insert word count here{]}
  }





\usepackage{amsthm}
\newtheorem{theorem}{Theorem}
\newtheorem{lemma}{Lemma}
\theoremstyle{definition}
\newtheorem{definition}{Definition}
\newtheorem{corollary}{Corollary}
\newtheorem{proposition}{Proposition}
\theoremstyle{definition}
\newtheorem{example}{Example}
\theoremstyle{remark}
\newtheorem*{remark}{Remark}
\begin{document}

\maketitle

\setcounter{secnumdepth}{0}



\section{Instructions}\label{instructions}

\begin{enumerate}
\def\labelenumi{\arabic{enumi}.}
\item
  Install
  \href{https://www.rstudio.com/products/rstudio/download/}{\texttt{R\ Studio}}
\item
  Install \href{https://github.com/crsh/papaja}{\texttt{papaja}} in
  \texttt{R\ Studio}
\item
  To convert this file into a Word document:

  \begin{enumerate}
  \def\labelenumii{\arabic{enumii}.}
  \tightlist
  \item
    make sure in the yaml block, \texttt{output} is set to
    \texttt{papaja::apa6\_word}
  \item
    click the \texttt{knit} button in \texttt{R\ Studio}
  \end{enumerate}
\item
  To convert this file into a PDF file:

  \begin{enumerate}
  \def\labelenumii{\arabic{enumii}.}
  \tightlist
  \item
    make sure in the yaml block, \texttt{output} is set to
    \texttt{papaja::apa6\_pdf}
  \item
    click the \texttt{knit} button in \texttt{R\ Studio}
  \end{enumerate}
\end{enumerate}

\section{In-Text Citation Examples}\label{in-text-citation-examples}

Citations are referred to by their citation keys (which you can specify
in your reference manager, like \emph{Mendeley}) in square brackets
(Magnuson et al., 2011), and multiple citations are separated by
semicolons like so (Magnuson, 2015; Magnuson et al., 2011).

You can add a prefix or suffix to a citation, for example, when you cite
pages from a book (Marr, 1982, pp. 24--27).

Although you can technically put the citation outside of square brackets
like Magnuson, Mirman, Luthra, Strauss, \& Harris (in press), you might
encounter formatting issues like unwanted ampersands. In cases like
this, you might want to suppress the author name(s) in the parenthesis
by adding \texttt{-} before the citation key, like this: Magnuson,
Mirman, Luthra, Strauss, and Harris (in press).

Also, the \textbf{(in press)} part of the previous reference is enabled
by my workaround \texttt{APA\_ML.csl} to work with \emph{Mendeley}.
Check out \texttt{example.bib} for \texttt{@Magnuson2018} to see how
it's set up.

For theses, dissertations, unpublished, and almost published
manuscripts, like (Li, 2016; Li et al., 2017; Li, Chang, Hung, \& Wu,
2014; Noordenbos, 2013), some care needs to be done when entering the
information in your reference manager. Check out \texttt{example.bib}
for these references to see the setup.

All cited references will be automatically generated at the end of the
converted document or under \texttt{\#\ References} /
\texttt{\#\ Bibliography}.

\section*{References}\label{references}
\addcontentsline{toc}{section}{References}

\hypertarget{refs}{}
\hypertarget{ref-Li2016}{}
Li, M. Y.-C. (2016). \emph{Subphonemic Sensitivity in Low Literacy
Adults}. (Master's thesis). University of Connecticut, Storrs, CT.

\hypertarget{ref-Li2017}{}
Li, M. Y.-C., Braze, D., Kukona, A., Johns, C. L., Tabor, W., Van Dyke,
J. A., \ldots{} Magnuson, J. S. (2017). \emph{Individual Differences in
Subphonemic Sensitivity and Reading-Related Abilities}. Manuscript
submitted for publication. \url{https://doi.org/10.17605/OSF.IO/3FNCY}

\hypertarget{ref-Li2014}{}
Li, M. Y.-C., Chang, E. C., Hung, D. L., \& Wu, D. H. (2014).
\emph{Functional MRI investigation of the effects of phonological and
visual similarity of English words on short-term memory in unbalanced
Chinese-English bilinguals}. Unpublished manuscript, National Central
University, Taiwan.

\hypertarget{ref-Magnuson2015}{}
Magnuson, J. S. (2015). Phoneme restoration and empirical coverage of
interactive activation and adaptive resonance models of human speech
processing. \emph{The Journal of the Acoustical Society of America},
\emph{137}(3), 1481--1492. \url{https://doi.org/10.1121/1.4904543}

\hypertarget{ref-Magnuson2011}{}
Magnuson, J. S., Kukona, A., Braze, D., Johns, C. L., Van Dyke, J. A.,
Tabor, W., \ldots{} Shankweiler, D. P. (2011). Phonological instability
in young adult poor readers: Time course measures and computational
modeling. In P. McCardle, B. Miller, J. R. Lee, \& O. J.-L. Tzeng
(Eds.), \emph{Dyslexia across languages: Orthography and the
brain-gene-behavior link} (pp. 184--201). Baltimore, Maryland: Paul H.
Brookes.

\hypertarget{ref-Magnuson2018}{}
Magnuson, J. S., Mirman, D., Luthra, S., Strauss, T., \& Harris, H. D.
(in press). Interaction in spoken word recognition models: Feedback
helps. \emph{Frontiers in Psychology}.
\url{https://doi.org/10.3389/fpsyg.2018.00369}

\hypertarget{ref-Marr1982}{}
Marr, D. (1982). \emph{Vision: A Computational Investigation into the
Human Representation and Processing of Visual Information}. San
Francisco: W. H. Freeman; Company.

\hypertarget{ref-Noordenbos2013a}{}
Noordenbos, M. W. (2013). \emph{Phonological representations in
dyslexia: Underspecified or overspecified?} (Doctoral dissertation).
Radboud University Nijmegen, The Netherlands. Retrieved from
\url{http://repository.ubn.ru.nl/bitstream/handle/2066/115718/115718.pdf?sequence=1}






\end{document}
